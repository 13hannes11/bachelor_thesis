\documentclass[12pt]{article}
\usepackage[utf8]{inputenc}
\usepackage[ngerman]{babel}
\usepackage{csquotes}

\title{Kollaborative Entscheidungsfindung im gruppenbasierten Konfigurationsprozess}
%\author{Hannes Fredrik Kuchelmeister}
\date{Oktober 2019}

%% --------------------------------
%% | Bibliography                 |
%% --------------------------------

\usepackage[citestyle=numeric,style=numeric,sorting=none, maxnames=2, giveninits=true, backend=biber]{biblatex}
\addbibresource{topic.bib}

%% --------------------------------
%% | Content                 |
%% --------------------------------

\begin{document}

\maketitle

\section{Motivation}

Produktkonfiguratoren unterstützen den Sales-Prozess im Unternehmen durch die Möglichkeit Produktwissen strukturiert zu pflegen und stellen den Vertriebsmitarbeitern ein Toolset zur Verfügung, mit dem valide Konfigurationen von Vertriebseinheiten garantiert werden. Typischerweise wird die Konfiguration von einem einzelnen Nutzer ausgeführt \cite{felferningGroupBasedConfiguration2016, velasquez-guevaraMultiSPLOTSupportingMultiuser2018a}.
Wenn es sich beim Konfigurationsprozess um komplexe Produkte handelt oder ein Szenario vorliegt in dem eine Gruppe beteiligt ist, steigt die Wahrscheinlichkeit, dass ein einzelner Nutzer falsche Entscheidungen trifft \cite{felfernigGroupDecisionSupport2011}. 
Beispielhafte Szenarien sind zum Beispiel Software Release Planungen, Urlaubsplanungen für Gruppen oder Planungen für den Bau von Gebäuden \cite{felfernigOpenConfiguration2014}. 
Bei diesen Szenarien handelt es sich um gruppenbasierte Konfigurationsprozesse, die in der Literatur \glqq Collaborative Configuration \grqq, \glqq Community Configuration \grqq oder auch \glqq Group-based Configuration \grqq genannt werden \cite{felferningGroupBasedConfiguration2016, felfernigOpenConfiguration2014,mendoncaCollaborativeProductConfiguration2008,felfernigKnowledgebasedConfigurationResearch2014}.

Im gruppenbasierten Konfigurationsprozess konfiguriert eine Gruppe kollaborativ ein Produkt. Das Ergebnis ist eine Konfiguration, die möglichst alle Präferenzen der Gruppenmitglieder berücksichtigt. Für die Umsetzung eines solchen Prozesses werden sowohl Konzepte für die Kollaboration, als auch für die Entscheidungsfindung benötigt. Dafür werden zusätzliche Funktionalitäten benötigt, wie beispielsweise Empfehlungssysteme zur Lösungsfindung oder Konfliktauflösung \cite{felfernigKnowledgebasedConfigurationResearch2014}. 
Erschwerend kommt hinzu, dass in Gruppen häufig eine dynamik oder auch eine hierarchie vorherrscht, die berücksichtig werden muss. Einzelne Mitglieder einer Gruppe sind beispielsweise nicht immer gleichberechtigt oder manche Entscheidungsprozesse nicht unbedingt demokratisch.

\section{Ziel der Arbeit}

Es soll Untersucht werden, welche Hilfsmittel sich eignen um den gruppenbasierten Konfigurationsprozess und damit die Entscheidungsfindung in Gruppen zu unterstützen. Weiterhin sollen die identifizierten Ansätze in einem holistischen und generischen Konzept zusammengeführt werden. Anhand der prototypischen und exemplarischen Umsetzung auf Basis des CAS Configurator Merlin der CAS Software AG und einem realen oder realitätsnahen Szenario soll evaluiert werden, wie gut sich diese Ansätze eignen, um Benutzer von Konfigurationssystemen bei der Gruppenkonfiguration zu unterstützen. 

\section{Forschungsfragen}

\begin{enumerate}
    \item Wie kann die Entscheidungsfindung von Gruppen im Konfigurationsprozess (durch Recommender-Systeme) unterstützt werden? (Leitfrage)
    \item Welche neuen Anforderungen an Konfigurationssysteme ergeben sich durch die kollaborative Entscheidungsfindung?
    \item Welchen Einfluss hat die Gruppenhierarchie oder -dynamik auf den Entscheidungsprozess in gruppenbasierten Konfigurationsszenarien?
\end{enumerate}

\section{Methodik}

Es wird „Design Science Research“ nach K. Peffers verwendet. Hierbei wird die Arbeit in sechs Aktivitäten unterteilt, nämlich: Problem Identifikation und Motivation, Zieldefinition für eine Lösung, Umsetzung, Demonstration, Evaluation und  Kommunikation \cite{peffersDesignScienceResearch2007}. Die Arbeit lässt sich Strukturell in mehrere Arbeitspakete aufteilen, wobei die Identifikation des Problems und relevanter Literatur an erster Stelle stehen. Im nächsten Schritt der Umsetzung wird mit Hilfe von weiterer Literatur eine Lösung des Problems erarbeitet. Am Ende steht die Evaluation und eine mögliche Verfeinerung der Problemstellung \cite{offermannOutlineDesignScience2009}.

\printbibliography[heading=bibintoc]
\end{document}