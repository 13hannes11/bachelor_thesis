\documentclass{article}
\usepackage[utf8]{inputenc}
\usepackage[english]{babel}

%% --------------------------------
%% | Quotation                 |
%% --------------------------------

\usepackage{csquotes}
\MakeOuterQuote{"}

\title{Collaborative Decision-Making in Group-Based Configuration Processes}
\author{Hannes F. Kuchelmeister}
\date{October 2019}

%% --------------------------------
%% | Bibliography                 |
%% --------------------------------

\usepackage[citestyle=numeric,style=numeric,sorting=none, maxnames=2, giveninits=true, backend=biber]{biblatex}
\addbibresource{topic.bib}

%% --------------------------------
%% | Content                 |
%% --------------------------------

\begin{document}

\maketitle

\section{Motivation}

Product configuration supports sales processes in companies through the possibility to structure and update knowledge about a product. A product configurator guarantees only valid product configurations are sold by a sales person. Usually, configuration processes are only done by a single user \cite{felferningGroupBasedConfiguration2016, velasquez-guevaraMultiSPLOTSupportingMultiuser2018a}.
In complex group configuration processes the risk of a single user making a mistake increases \cite{felfernigGroupDecisionSupport2011}.
Example scenarios are software release planning, travel planning, and construction planning \cite{felfernigOpenConfiguration2014}. 
All of these scenarios are group based configuration processes which are called "collaborative configuration", "community configuration", or "group-based configuration" \cite{felferningGroupBasedConfiguration2016, felfernigOpenConfiguration2014,mendoncaCollaborativeProductConfiguration2008,felfernigKnowledgebasedConfigurationResearch2014}.

In a group based configuration processes a group of people collaboratively configures a product which results in a configuration that ideally includes all preferences of participants. Implementation of such processes build upon concepts for collaboration, decision-making and require additional functionality such as recommender systems for finding solutions and conflict resolution \cite{felfernigKnowledgebasedConfigurationResearch2014}.

%TODO: Research gap and previous work -> are there many existing methods?

Furthermore, groups often have hierarchies or split competences that need to be considered. Moreover, decision processes in groups are not always democratic and individual members can have different roles.

\section{Objectives}

It will be investigated which methods and tools exist and are suitable to aid in group based configuration processes to support decision-making in these contexts. Identified methods and tools will be used in a holistic and generic approach. 
The implementation will be done on the basis of \emph{CAS Configurator Merlin \copyright} of \emph{CAS Software AG}. The suitability to aid users in configuration processes with this approach will be evaluated based on realistic scenarios. In addition, possible limitations and improvement potential will be discussed.

\section{Research Questions}

\begin{enumerate}
    \item How can decision-making in group configuration processes be supported (through recommender systems)? (central question)
    \item Which requirements do configuration systems for collaborative decision-making have?
    \item How does group hierarchy and structure influence the decision-making process in group based configuration scenarios?
\end{enumerate}

\section{Methods}

"Design Science Research" is used based on the definitions of K. Peffers et al. The thesis is divided into six activities: "Problem Identification and Motivation", "Objectives of a Solution", "Design and Development", "Demonstration", "Evaluation", and "Communication" \cite{peffersDesignScienceResearch2007}. 
These steps are grouped into three self-contained work packages \cite[p.~6]{offermannOutlineDesignScience2009}.

\subsection{Problem Identification}

\subsection{Design and Development}

\subsection{Demonstration and Evaluation}

\printbibliography[heading=bibintoc]
\end{document}