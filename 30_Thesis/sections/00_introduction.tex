\chapter{Introduction}
\label{ch:Introduction}

\section{Motivation}
\label{sec:Introduction:Goals}

Customers and sales people used to be faced with big catalogues that describe how a complex product can be build. These catalogues specified what is possible to combine but it was easy to make mistakes. The sales process took a long time because orders had to be manually validated. With the advent of mass customization these issues started to be addressed. In mass customization a configurator is used. The configurator has a rulebook that contains all the attributes of a product, their corresponding characteristics and rules on how these can be combined. A system like that allows to reduce the workload and cost of sales \cite{shafieeCostBenefitAnalysis2018}. "[P]roduct configuration is seen as a team activity with divergent interests" \cite{mendoncaCollaborativeProductConfiguration2008} and therefore research in the field of group-based configuration started receiving more attention. 
To give an idea of situations that can use group-based configuration, here are some examples:
\begin{itemize}
    \item A company's lorry fleet (e.g. driver, purchasing-manager, marketing manager)
    \item A company's customer management software system (e.g. salesperson, human resource manager, accounting manager)
    \item A public project to get a new area at a zoo (e.g. visitor, director, animal keeper)
    \item Managing a forest (e.g. owner, environmental protection agency, consumer representative)
    \item An existing company building and how it should be furnished (e.g. landlord, employee representative, CEO)
\end{itemize}

Unfortunately, making decisions in the group comes with problems as a lack of communication leads to worse decision outcomes \cite{atasItemRecommendationUsing2017}. Group dynamics and biases can lead to suboptimal decisions \cite{kerrBiasJudgmentComparing1996}.
Generally group decisions are complex and often the process that yields the decision result is unstructured, thereby not providing any reproducibility of the success. Groups have different power structures and usually individuals have different interests. Moreover, finding solutions is a rather complex task and group decisions can suffer intransparency.

Group recommenders promise to help with that as they can take individual user preferences and find good compromises for the whole group. They are used in movies, music and travel \cite{garciaGroupRecommenderSystem2009, piliponyte2013sequential, peraGroupRecommenderMovies2013,felfernigGroupRecommenderApplications2018}. The existing literature on recommenders for groups is extensive with many different approaches and domains \cite{delicResearchMethodsGroup2016, chenInterfaceInteractionDesign2011, atasItemRecommendationUsing2017, jamesonRecommendationGroups2007, chenEmpatheticonsDesigningEmotion2014, liuCGSPAComprehensiveGroup2019} but to date there have not been any approaches to combine them with group-based configuration. There have been approaches to combine recommendation techniques with configuration but these were limited to configuration for a single user only \cite{pereiraFeatureBasedPersonalizedRecommender2016, scholzConfigurationbasedRecommenderSystem2017, scholzEffectsDecisionSpace2017}.


The above listed examples of use cases for group-based configuration show that ordinary group recommenders cannot be used here as they, unlike configuration, operate on simple products. Configuration on the other hand operates with solutions that are successively built by combining standardized features. This leads to three problems when trying to apply recommenders in this field: first, because of simple combinatorics, the number of variants is enormous. Second, because of the interrelation between features, a small change in configuration can lead to many subsequent changes and thus a completely different product. Third, during configuration there is an already pre-existing configuration state that potentially is expensive to change. This could be the current state of a situation for example how a building currently looks like, an already made decisions on a project that would be expensive to change or simply an already agreed part that was decided on earlier in the project stage. The group recommender needs to account for these behaviours. Therefore it is necessary to explore new ideas to take knowledge from group recommendation into account.

\section{Goals of this Thesis}
\label{sec:Introduction:Goals}

This thesis aims to show the viability of using group recommenders in a configuration setting by producing and evaluating a prototype. It is discussed what needs to be done to adapt group recommenders to allow usage of the basic recommendation technique of item-based recommendation. The research question for this thesis is the following.

\begin{itemize}
    \item How can a group recommender translate individual preferences into recommendations that improve the overall satisfaction of group members while considering constraints given by the configuration state? 
\end{itemize}


\section{Structure of this Thesis}
\label{sec:Introduction:Structure}

This thesis first will given an introduction, then present related work. Next the concept for a recommender is presented. Afterwards design and implementation of the prototype, produced in this thesis, are discussed, followed by its evaluation. Last, a conclusion is made which includes a summary and further research possibilities.
