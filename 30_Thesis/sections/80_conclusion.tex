\chapter{Conclusion}
\label{ch:Conclusion}

This chapter gives a summary about the thesis, discusses limitations and gives and outlook on possible further research.

\section{Summary}
\label{sec:Conclusion:Summary}

\todo[inline]{hab da nochmal drüber nachgedacht: die Zusammenfassung ist mir noch zu knapp.
hier solltest du nochmal den Bogen ganz an den Anfang schlagen: was war die Motivation \& welche Forschungsfragen hattest du gestellt? wurden die Fragen beantwortet und wie?
tatsächlich solltest du hier nochmal kurz \& knapp durch die komplette Thesis führen (stell dir vor deine Profs lesen nur dieses Kapitel (und das ist sehr wahrscheinlich) -> sie sollen trotzdem verstehen, was du alles gemacht hast, wenn auch nur auf einem top-level). Also:
-welche theoretischen Grunlagen wurden aufbereitet
-was war das konzept
-wie wurde implementiert
- wie wurde evaluiert und mit welchem Ergebnis
ja, das wiederholt sich, muss hier aber so sein.}




\begin{itemize}
    \item Motivation
        \begin{itemize}
            \item group decisions hard
            \item products not always single user configuration
            \item group avoids mistakes
            \item compromise found
            \item reproducible
        \end{itemize}
    \item Forschungsfrage
        \begin{itemize}
            \item show viability of group recommenderrs for configuration
            \item How can a group recommender translate individual preferences into recommendations that improve the overall satisfaction of group members while considering constraints given by the configuration state?
        \end{itemize}
    \item Theoretische Grundlagen
        \begin{itemize}
            \item product configuration
            \item group-based configuration
            \item recommender systems
                \begin{itemize}
                    \item Collaborative Filtering
                    \item Content-Based Filtering 
                    \item Constraint based recommendation
                \end{itemize}
            \item group recommmender
        \end{itemize}
    \item konzept
        \begin{itemize}
            \item all users configure simultaniously
            \item recommendations are generated based on a pool of known valid configurations (database)
            \item scoring and penalty function
        \end{itemize}
    \item implementiert
        \begin{itemize}
            \item extending configurator merlin
            \item microservice open source
        \end{itemize}
    \item was wurde evaluiert und welches ergebnis?
        \begin{itemize}
            \item introduce satisfaction offline metric
            \item Finetune and select parameter for this metric
            \item Look at homogeneous hetereogeneous and random groups of size four
            \item Results: works already with just part configuration. Especially great effects for hetereogeneous and random. For homogenous groups reduced knowledge yielded bad results
        \end{itemize}
\end{itemize}



The aim of this thesis was to look at recommendation approaches and to propose a concept for recommenders in group-based configuration which helps groups with forming compromises. 

The proposed concept suggests means to calculate per-user-scores for configurations. This allows the usage of a basic approach like item-based recommendation. 

Moreover, these recommendations help a group to form a decision and show possible compromises for groups without requiring direct communication between group members. The usefulness of this approach was validated by introducing an offline metric to model group member satisfaction. 

Generated recommendations improve group decisions and thereby the implemented recommender can be used as a solid base line to improve upon. 

Overall it is possible to suggest compromises in group-based configuration which are based on group member's preferences. 

Furthermore, the recommender works well, also, when knowledge is limited to only a subset of the configuration space. 

Moreover, it is possible to conclude that content-based recommendation approaches can be adapted to be used for group-based configuration settings.

\section{Limitations and Further Research}
\label{sec:Conclusion:LimitationsFurtherResearch}

Due to the scope of this thesis it was not possible to analyse all possible scenarios for the proposed approach. This section describes the limitations and how they can be overcome.

An offline satisfaction metric is introduced in this thesis and it could lead to results that differ from real life impressions of people. It has to yet be validated. The validations of this metric allows the use of another metric for scenarios that lack a suitable existing metrics. Moreover, understanding ´the relation between the introduced metric and actual satisfaction can help to form more accurate satisfaction models. This helps to understand finding better compromises and such a metric can directly be used as a component of a group recommender.

In this thesis only one use case was considered. Therefore, different use cases could yield different results. It is unclear if different scenarios that either are more complex or greater in size yield the same results. The successful application to larger sized products could help for large complex projects that have multiple experts from also differing areas that do not necessarily agree. Also, larger products potentially reach limitations of the recommender as the solution space grows quickly. An approach that clusters configurations and other means of optimization can help with performance. Complex and large products also might lead to usability issues as users potentially are overwhelmed with choices and setting all preferences individually might be infeasible. Therefore indirect means of acquiring preferences should be incorporates for more complex and larger configurations. 

Furthermore, groups were automatically generated and thus, possibly lack resemblance to common real world group constellations. Therefore, real world group constellations for decisions should be analysed. This allows the validation of the used groups and can lead to more useful understanding for generating synthetic groups that resemble actual groups. Thus, less time has to be spend in acquiring user data.

Moreover, the recommender performed not ideal with homogenous groups especially when knowledge about the solution space was limited. Therefore, methods of detecting homogenous groups could detect cases in which the recommenders perform badly and use other recommenders instead.

Additionally, the group size was fixed to four members and deviating results for differently sized groups are possible. Moreover, this approach can be extended to potentially allow a whole community of hundreds of people to decide about neighbourhood changes. This could range from the layout of a new community centre to staffing, equipment and uses. Therefore, such approaches of group-based configuration can be used for public participation in projects. Helping communities to build trust and be involved more involved in decisions.

Finally, the approach used in this thesis assumes a flat group hierarchy. Modelling knowledge and hierarchy of a group can help to improve group decisions further as supervisors do not feel overrun by their employees and knowledge of experts in certain parts of a product or solution can use that knowledge to guide the decision that area. Experts in other areas could have more say in areas of their expertise. Therefore, decisions could be expert and hierarchy driven which should help with group satisfaction about compromises.