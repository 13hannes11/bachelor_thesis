\chapter{Conclusion}
\label{ch:Conclusion}

This chapter gives a summary about the thesis, discusses limitations and gives and outlook on possible further research.

\todo[inline]{write chapter}

\section{Summary}
\label{sec:Conclusion:Summary}

The aim of this thesis was to look at recommendation approaches and to propose a concept for recommenders in group-based configuration which helps groups with forming compromises. The proposed concept suggests means to calculate per user scores for configurations. This allows the usage of a basic approach like item-based recommendation. Moreover, these recommendations help a group to form a decision and show possible compromises for groups without requiring a lot of communication between group members. The usefulness of this approach was validated by introducing an offline metric to model group member satisfaction. Generated recommendations improve group decisions and thereby the implemented recommender can be used as a solid base line to improve upon. 
Overall it is possible to suggest compromises in group-based configuration which are based on group member's preferences. Furthermore, the recommender works well, also, when knowledge is limited to only a subset of the configuration space. Moreover, it is possible to conclude that content-based recommendation approaches can be adapted to be used for group-based configuration settings.

\section{Limitations}
\label{sec:Conclusion:Limitations}

Due to the scope of this thesis it was not possible to analyse all possible scenarios for the proposed approach. The findings need to be validated with real users as only an offline metric has been used. Moreover, only one use case was considered. Therefore, different use cases could yield different results. Furthermore, groups were automatically generated and thus, possibly lack resemblance to common real world group constellations. Additionally, the group size was fixed to four members and deviating results for differently sized groups are possible. Finally, larger product that contain many more features and many more characteristics potentially can limit this approach.

\section{Further Research}
\label{sec:Conclusion:PossibleExtensions}

During the work of this thesis several new research possibilities came up. This section will name and discuss them.
First, further investigation can be done on how the approach of storing parts of the solution space behaves with larger products with a larger solution space and how performance can be optimised. Moreover, other methods of gathering preferences can be designed. A user might not need to enter all his preferences but only the most important once. Other ways of inferring preferences might be an area to look at, too. Moreover, the influence of different hierarchies in groups could be taken into account for decisions. This can be in terms of positional hierarchy but also in terms of knowledge about specific parts of a product.
Additionally, research can be undertaken to validate the offline satisfaction metric used in this thesis. How close does it correlate with actual user preferences?
Another possible extension area to look at is the identification of too homogenous groups as this was the weak point of the recommender and how this knowledge can be used to improve decision making. More complex products and different scenarios are another area to undertake research in. Last, research in validating the performance of the recommender in a real group settings and comparing it to group meetings without the aid of a recommender should be undertaken.