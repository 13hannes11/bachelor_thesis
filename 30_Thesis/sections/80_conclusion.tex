\chapter{Conclusion}
\label{ch:Conclusion}

This chapter gives a summary about the thesis, discusses limitations and gives and outlook on possible further research.

\todo[inline]{write chapter}

\section{Summary}
\label{sec:Conclusion:Summary}

The aim of this thesis was to look at recommendation approaches and to propose a concept for recommenders in group-based configuration which helps groups with forming compromises. The proposed concept suggests means to calculate per user scores for configurations. This allows the usage of a basic approach like item-based recommendation. Moreover, these recommendations help a group to form a decision and show possible compromises for groups without requiring a lot of communication between group members. The usefulness of this approach was validated by introducing an offline metric to model group member satisfaction. Generated recommendations improve group decisions and thereby the implemented recommender can be used as a solid base line to improve upon. 
Overall it is possible to suggest compromises in group-based configuration which are based on group member's preferences. Furthermore, the recommender works well, also, when knowledge is limited to only a subset of the configuration space. Moreover, it is possible to conclude that content-based recommendation approaches can be adapted to be used for group-based configuration settings.

\section{Limitations}
\label{sec:Conclusion:Limitations}
\begin{itemize}
    \item Recognising the Limitations
        \begin{itemize}
            \item Only looked at one use case
            \item Only offline evaluation
            \item Groups automatically generated - No real groups
            \item Group size fixed to four people
            \item Performance for bigger products not validated
        \end{itemize}
    \item Making recommendations for further research work 
\end{itemize}

\section{Further Research}
\label{sec:Conclusion:PossibleExtensions}

\begin{itemize}
    \item How to optimise such that no need to search through all stored finished configurations is necessary? Something like tree like structure to cluster elements
    \item How to model hierarchy and knowledge about product components in preferences?
    \item Letting users set preferences for product functions (e.g. for a forest a recreation function, a productive function, a protective function, etc.). How does it compare to explicitly choosing preferences?
    \item Does the assumption that the closer the configuration state is to a finished configuration, the less the satisfaction increase and the less difference among recommended configurations hold true?
    \item Validating if satisfaction correlates with theoretical metric used in this thesis
    \item Identification of too homogenous groups to use single person recommender.
    \item Test more complex products with more attributes and characteristics. Do they see the same effect in regards to stored configuration and recommendation quality.
    \item Evaluate different types of generating user score for configuration
    \item Larger Groups
    \item Modelling hierarchy and knowledge in group decisions for configuration
    \item Approaches towards configuration that reduce complexity and guide users for setting preferences
    \item Implicitly getting preferences
\end{itemize}
