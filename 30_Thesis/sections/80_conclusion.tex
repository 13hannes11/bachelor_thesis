\chapter{Conclusion}
\label{ch:Conclusion}

\todo[inline]{write chapter}

\begin{itemize}
    \item Restating the aims of the study
\end{itemize}

\section{Summary}
\label{sec:Conclusion:Summary}
\begin{itemize}
    \item Summarising main research findings
    \item Suggesting implications for the field of knowledge
    \item Explaining the significance of the findings or contribution of the study
\end{itemize}

\section{Limitations}
\label{sec:Conclusion:Limitations}
\begin{itemize}
    \item Recognising the Limitations
    \item Making recommendations for further research work 
\end{itemize}

\section{Further Research}
\label{sec:Conclusion:PossibleExtensions}

\begin{itemize}
    \item How to optimise such that no need to search through all stored finished configurations is necessary? (e.g. improve runtime from $\mathcal{O}(n)$ to $\mathcal{O}(log\ n)$)
    \item How to model hierarchy and knowledge about product components in preferences?
    \item Letting users set preferences for product functions (e.g. for a forest a recreation function, a productive function, a protective function, etc.). How does it compare to explicitly choosing preferences?
    \item Does the assumption that the closer the configuration state is to a finished configuration, the less the satisfaction increase and the less difference among recommended configurations hold true?
    \item Validating if satisfaction correlates with theoretical metric used in this thesis
    \item Identification of too homogenous groups to use single person recommender.
    \item Test more complex products with more atrributes and characteristics. Do they see the same effect in regards to stored configuration and recommendation quality.
    \item Larger Groups
\end{itemize}
