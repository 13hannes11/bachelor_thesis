\chapter{Conclusion}
\label{ch:Conclusion}

This chapter summarises the thesis and describes the exact steps that were undertaken, starting with an overview of the foundations regarding group recommenders and configuration. Furthermore, a concept for a recommender for group-based configuration is introduced, implemented as prototype and evaluated. Last, limitations of this thesis and the recommender system are presented and further research is proposed.

\section{Summary}
\label{sec:Conclusion:Summary}


This thesis was motivated by the research area of group-based configuration gaining more interest. As group decisions come with many problems and biases, recommender systems are used  to help with group decisions. This avoids mistakes and helps with reproducibility of successful group decisions. To date, no research has been carried out on recommendations regarding group-based configurations. This is why the research question of this thesis focuses on: "How can a group recommender translate individual preferences into recommendations that improve the overall satisfaction of group members while considering constraints given by the configuration state?". This thesis answers the research questions by proposing a concept, implementing it as a prototype and evaluating it. Thereby the viability of such a system and such an approach is shown.
First, the thesis introduces foundations of product configuration and extends them to group-based product configuration. Next, recommender systems are introduced and three basic approaches, collaborative filtering, content-based filtering and constraint based recommendation, are compared. Last, the \hyperref[ch:Foundations]{foundations chapter} allots an introduction into group recommendation.
Following\autoref{ch:Foundations} (Foundations), a concept for an item-based recommender for group-based configuration, is introduced. This concept uses a database of finished configurations to choose a configuration that fits the group best. Each user's preferences are used to assign a score to a configuration. The score of all group members is then aggregated into a group score and the best configuration from the database is recommended. The recommender also all penalties for deviating from the current configuration state.
Later, the concept is implemented as an open source microservice which is integrated into an already existing group-based configurator. The source code is available at \cite{kuchelmeister13hannes11BachelorThesis}.
Last, an offline metric for satisfaction is introduced and it is used for evaluation. Three group types are evaluated, homogenous groups, random groups and heterogeneous groups. Overall, the evaluation shows that the recommender yields good results for groups and helps groups to form a compromise. Satisfaction among group members increases overall. A simple item-based approach therefore  already improves group decisions by finding good compromises. This is also the case when the knowledge of the recommender is limited.

\section{Limitations and Further Research}
\label{sec:Conclusion:LimitationsFurtherResearch}

Due to the scope of this thesis it was not feasible to analyse all possible scenarios for the proposed approach. This section describes the limitations of the thesis and how they can be overcome.

An offline satisfaction metric is introduced in this thesis which may lead to results that differ from real life impressions of people. It has yet to be validated in a real-world setting. The validations of this metric allows the use of the introduced satisfaction metric for other scenarios that also lack suitable existing metrics. Moreover, understanding the relation between the introduced metric and actual satisfaction can help to form more accurate satisfaction models. This helps to understand how to find better compromises and, at the same time, such a metric can directly be used as a component of a group recommender.

In this thesis only a single use case was examined. Therefore, different use cases could yield different results in terms of satisfaction and the tolerance of limited knowledge. It is unclear if different scenarios which are either more complex or have more features and characteristics yield the same results. The successful application to larger products could help with large complex projects that include multiple experts from differing areas that do not necessarily agree. Also, larger products potentially reach limitations of the recommender as the solution space grows quickly. An approach that clusters configurations and other means of optimization can help with performance. Complex and large products might lead to usability issues as users potentially are overwhelmed with choices and setting all preferences individually might be infeasible. Thus, indirect means of acquiring preferences should be incorporated for more complex and larger configurations. 

Furthermore, groups were automatically generated and, thus, possibly a lack of resemblance to common real-world group constellations. Consequently, real world group constellations for decisions should be analysed. This will allow the validation of the implemented groups and can lead to a better understanding of generating synthetic groups that resemble actual groups. Thus, less time has to be spent in acquiring user data.

Moreover, the recommender did not perform ideally with homogenous groups, especially when knowledge about the solution space was limited. Hence, methods of detecting homogenous groups could detect cases in which the recommender perform poorly and use other recommenders instead. A recommender that should be evaluated in this context is presented by \citeauthor{choudharyMulticriteriaGroupRecommender2020} \cite{choudharyMulticriteriaGroupRecommender2020}. This recommender fares better with homogeneous groups opposed to random groups.

Additionally, the group size was fixed to four members and because \citeauthor{choudharyMulticriteriaGroupRecommender2020} note that their recommender does better in smaller groups, deviating results for differently sized groups are possible.Moreover, this approach can be extended to potentially allow a whole community of hundreds of people to decide about neighbourhood changes. This could range from the layout of a new community centre, staffing, equipment and uses. Therefore, such approaches of group-based configuration can be used for public participation in projects helping communities to build trust and be more involved in decisions.

Finally, the approach used in this thesis assumes a flat group hierarchy. Modelling knowledge and hierarchy of a group can help to improve group decisions further as supervisors do not feel overrun by their employees and knowledge of experts on certain parts of a product or solution can use that knowledge to guide the decision that area. Experts in other areas could have more say in areas of their expertise. Thus, decisions could be expert and hierarchy driven which should help with group satisfaction about compromises.