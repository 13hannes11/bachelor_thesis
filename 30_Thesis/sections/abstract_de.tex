\Abstract

Gruppenbasierte Konfiguration ist ein relativ neues Feld, das sich mit der Konfiguration von Produkten, Dienstleistungen und Lösungen in Gruppe auseinandersetzt. Das Treffen guter Gruppenentscheidungen ist jedoch komplex und nicht immer reproduzierbar. Deshalb gibt es Empfehlungssysteme für Gruppen. Diese wurden jedoch noch nicht für Gruppenkonfiguration eingesetzt. Daher wird in dieser Thesis ein Konzept zur Integration eines Item-Basierten-Gruppenempfehlungssystems in einen Gruppen-Konfigurator vorgestellt, prototypisch implementiert und evaluiert. Das Konzept zeigt auf, wie Präferenzen einzelner Gruppenmitglieder kombiniert werden, um eine Gruppenbewertung für eine Konfiguration zu erhalten. Dieser Ansatz wird nun genutzt, um die durch die Gruppe höchstbewertete Konfiguration aus einer Datenbank auszuwählen und zu empfehlen. Die Evaluation mit synthetischen Gruppen mit vier Personen erzielt gute Ergebnisse, besonders bei heterogenen Gruppen.