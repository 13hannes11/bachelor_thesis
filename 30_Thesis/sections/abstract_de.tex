\Abstract

Gruppenbasierte Konfiguration ist ein noch relativ junges Feld, das sich mit der Konfiguration von Produkten, Dienstleistungen und Lösungen durch Gruppen auseinandersetzt und Gruppenentscheidungen unterstützt. Gute Gruppenentscheidungen zu treffen ist jedoch komplex und nicht immer reproduzierbar. Empfehlungssysteme für Gruppen können hier Abhilfe schaffen. Diese wurden jedoch noch nicht für gruppenbasierte Konfiguration eingesetzt. In dieser Thesis wird ein Konzept zur Integration eines item-basierten Gruppenempfehlungssystems in einen Gruppen-Konfigurator vorgestellt, prototypisch implementiert und evaluiert. Das Konzept zeigt auf, wie Präferenzen einzelner Gruppenmitglieder kombiniert werden können, um eine Gruppenbewertung für eine Konfiguration zu erhalten. Dieser Ansatz wird genutzt, um die durch die Gruppe höchstbewertete Konfiguration aus einer Datenbank auszuwählen und zu empfehlen. Die Evaluation mit synthetischen Gruppen mit vier Personen erzielt gute Ergebnisse, insbesondere bei heterogenen Gruppen.
