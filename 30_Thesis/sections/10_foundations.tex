\chapter{Foundations}
\label{ch:Foundations}

This chapter gives an overview over the foundations this thesis is based upon.
\todo[inline]{write introduction paragraph which tells the order things are introduced in.}

\section{Product Configuration}
\label{sec:Foundations:ProductConfiguration}

Product configuration is a process consisting of a series of decision tasks whereby a product is constructed of components which interact with each other. During a configuration process no new components are created. Their interplay and specification is defined beforehand \cite[~ pp. 42, 43]{sabinProductConfigurationFrameworksa1998}. This section defines product configuration analogous to \citeauthor{felfernigOpenConfiguration2014} \cite{felfernigOpenConfiguration2014}.

Formally a configuration problem can be specified as a \emph{constraint satisfaction problem (CSP)} \cite{tsangFoundationsConstraintSatisfaction1993} as 
\begin{equation} \label{eq:Foundations:ProductConfiguration:ConstraintSatisfactionProblem}
    CSP(V,\mathfrak{D},C),
\end{equation}
where $V$ is a set of \emph{variables} (which in this thesis will also be referred to as \emph{features}) with
\begin{equation} \label{eq:Foundations:ProductConfiguration:Variables}
    V = \{v_1, \dotsc, v_m\},
\end{equation}
a \emph{domain mapping} $\mathfrak{D}$ that maps variable to their corresponding domain values $D$ (which in this thesis will be also referred to as \emph{characteristics}) i.e. the values that can be assigned to each variable
\begin{equation}\label{eq:Foundations:ProductConfiguration:DomainMapping}
    \mathfrak{D} : V \to D; x \mapsto \mathfrak{D}(x) \qquad where \ D = \{d_1, \dotsc, d_o\},
\end{equation}
and \emph{constraints} $C$ that limit the solution space with 
\begin{equation}\label{eq:Foundations:ProductConfiguration:Constraints}
    C = \{c_1, \cdots, c_k\}.
\end{equation}

In group-based configuration (also known as collaborative or group configuration) a group instead of a single user is set to configure a configuration. This entails challenges in terms of synchronising  workspaces and keeping the data consistent for every group member. \citeauthor{raabKollaborativeProduktkonfigurationEchtzeit2019}'s \cite{raabKollaborativeProduktkonfigurationEchtzeit2019} approach, which this thesis extends, is to treat the group configuration the same as one shared configuration and to sync the selection of attributes across clients.

\subsection{Group-Based Product Configuration}
\label{sec:Foundations:GroupBasedProductConfiguration}

\todo[inline]{write here about group based configuration and challanges posed by it}


\section{Recommender System}
\label{sec:Foundations:RecommenderSystem}

A recommender system is a system that gives individualized recommendations to users to guide them through a large space of objects \cite[~ p. 331]{burkeHybridRecommenderSystems2002}.

There are several approaches to recommender systems presented in \cite{felfernigGroupRecommenderSystems2018}, some of them are: collaborative filtering, constraint-based recommendation, content-based filtering and hybrid recommendation.

\begin{table}
    \centering    
    \begin{tabular}{ l | c | c | c | c | c }
        & The Matrix & Titanic & Die Hard & Forest Gump & Wall-E \\ \hline
         John  & 5 & 1 &   & 2 & 2 \\
         Lucy  & 1 & 5 & 2 & 5 & 5 \\
         Eric  & 2 & ? & 3 & 5 & 4 \\
         Diane & 4 & 3 & 5 & 3 &   \\
    \end{tabular}
    \caption{An example showing users ratings for movies by \citeauthor{ningComprehensiveSurveyNeighborhoodBased2015} \cite{ningComprehensiveSurveyNeighborhoodBased2015}.}
    
    \label{tab:Foundations:RecommenderSystem:MoviePreferences}
\end{table}

\subsection{Collaborative Filtering}
In collaborative filtering a users rating for unknown items is predicted by finding similar users who have rated it. Their rating is used as prediction
\cite[~ pp. 7, 8]{felfernigDecisionTasksBasic2018}.

Collaborative Filtering can not only be done using users, it can also be item-based. Hereby the similarity between items is used for a recommendation and not similar users \cite{ricciRecommenderSystemsHandbook2015}. In the context of configuration the similarity to other historic configurations can be used which makes it an item based approach. 

\autoref{tab:Foundations:RecommenderSystem:MoviePreferences} shows an example rating matrix. A simple user-based way to calculate a rating would be to use a k-nearest neighbour (kNN) algorithm and then take the average of those ratings. Using this method with $k := 2$ and euclidean distance our closest neighbours are \textit{Lucy} and \textit{Diane} therefore giving us a predicted rating of $4$. If an item-based approach is used instead, it will be tried to find similar items based on the user's rating. An example of similar items here would be \textit{Forest Gump} and \textit{Wall-E} as John and Lucy each have given them the sane rating and Eric's rating is off by one. Using again kNN with $k := 2$ it is found that \textit{Forest Gump} and \textit{Wall-E} are the most similar to \textit{Titanic} thereby having a predicted rating of $4.5$.
However this simple similarity and prediction function does not take into account different distances. For example Lucy's ratings are more similar compared to Eric's than Diane's but Diane's and Lucy's rating is valued the same amount.

\subsection{Constraint-Based Recommendation}
Hereby filter rules are defined which filter out items that don't fulfil specified rules. A user models their requirements with these rules and thereby gets a list of recommended items. This approach requires deep knowledge about a product because it needs a detailed description of features  \cite[~ p. 12]{felfernigDecisionTasksBasic2018}.

Our movie example (see \autoref{tab:Foundations:RecommenderSystem:MoviePreferences}) requires additional information for example about plot structure, pacing, length and other attributes of the movie. Now the user could enter filtering criteria, e.g. that the movie should be no longer than 120 minutes, be categorized as action or thriller and have a fast pacing. The system will only recommend movies that fit into these categories.

\subsection{Content-Based Filtering}
For the content-based filtering approach, items and users are assigned to categories. Based on consumption and rating of items a user will have implicit ratings for categories. Predictions are now made based on a categories of the new item \cite[~ pp. 10, 11]{felfernigDecisionTasksBasic2018}.

Using the example from \autoref{tab:Foundations:RecommenderSystem:MoviePreferences} and using an additional category matrix (see \autoref{tab:Foundations:RecommenderSystem:ContentBasedFilteringCategories}) it a rating matrix per category can be derived (using the average rating of the user of each movie contained in this category). The result can be seen in \autoref{tab:Foundations:RecommenderSystem:ContentBasedFilteringProfiles}. To predict Eric's rating of Titanic, the categories of \textit{Titanic} and averages of Eric's implicit rating per category are used. Titanic is only in the category romance and as Eric's rating of \textit{Forest Gump} is $5$ the prediction is a rating of $5$. Categories don't have to be the genre, they could be any kind of data about a movie.

\begin{table}
    \centering    
    \begin{tabular}{ l | c | c | c | c | c }
        & The Matrix & Titanic & Die Hard & Forest Gump & Wall-E \\ \hline
         Action  & x &  & x  &  &  \\
         Sci-Fi  & x &  &  &  &  \\
         Thriller  &  & & x &  &  \\
         Romance & & x & & x & \\
         Family & & & & x & x \\
    \end{tabular}
    \caption{Showing example categories for movies in \autoref{tab:Foundations:RecommenderSystem:MoviePreferences}.}
    
    \label{tab:Foundations:RecommenderSystem:ContentBasedFilteringCategories}
\end{table}

\begin{table}
    \centering    
    \begin{tabular}{ l | c | c | c | c | c }
        & Action & Sci-Fi & Thriller & Romance & Family \\ \hline
        John  & 5 & 5 & & 1.5 & 2 \\
        Lucy  & 1.5 & 1 & 2 & 5 & 5 \\
        Eric  & 2.5 & 2 & 3 & 5 & 4.5 \\
        Diane & 4.5 & 4 & 5 & 3 & 3  \\
    \end{tabular}
    \caption{User profiles generated from categories and rating from \autoref{tab:Foundations:RecommenderSystem:MoviePreferences} and \autoref{tab:Foundations:RecommenderSystem:ContentBasedFilteringCategories}.}
    
    \label{tab:Foundations:RecommenderSystem:ContentBasedFilteringProfiles}
\end{table}

\subsubsection{Advantages over Collaborative Filtering}
\todo[inline]{write as text and not bullet points}
\begin{itemize}
    \item No cold start problem for items
    \item No grey sheep problem as not dependent on similar groups having existed before.
    \item Domain knowledge is existent
    \item No issues with data sparsity as item description is given by product structure
    \item No reliance on preferences that would result in a comparison space that is too large
    \item No dependence of historic group preference accuracy 
\end{itemize}

\subsubsection{Advantages over Constrained-Based Recommendation}
\todo[inline]{write as text and not bullet points}
\begin{itemize}
    \item Configuration state does not cause absence of recommendations
    \item Expendable to also support constraints 
    \item No need to handle inconsistencies explicitly
\end{itemize}


\begin{table}
    \begin{center}
        \begin{tabularx}{\columnwidth}{X|X|X}
            & advantages & disadvantages \\
            \hline
            Collaborative Filtering 
            &   \begin{itemize}[noitemsep,topsep=0pt,parsep=0pt,partopsep=0pt, leftmargin=3.5mm]
                    \item Serendipity of results 
                    \item Automatic learning of market segments
                    \item No domain knowledge required
                \end{itemize}
            &   \begin{itemize}[noitemsep,topsep=0pt,parsep=0pt,partopsep=0pt, leftmargin=3.5mm]
                    \item Cold start problem for users and items
                    \item Grey sheep problem
                    \item Quality based on rating quality
                    \item Data sparsity
                    \item Privacy not guaranteed
                \end{itemize} \\
            \hline
            Content-Based Filtering 
            &   \begin{itemize}[noitemsep,topsep=0pt,parsep=0pt,partopsep=0pt, leftmargin=3.5mm]
                    \item No community required 
                    \item User independent
                    \item Transparent
                    \item No item cold start
                    \item Simplicity
                    \item Robust
                    \item Stable to constant influx of new users
                \end{itemize}
            &   \begin{itemize}[noitemsep,topsep=0pt,parsep=0pt,partopsep=0pt, leftmargin=3.5mm]
                    \item Overspecialisation
                    \item No serendipity
                    \item User cold start problem
                    \item Requires domain knowledge
                \end{itemize} \\
            \hline
            Constraint-Based Recommendation 
            &   \begin{itemize}[noitemsep,topsep=0pt,parsep=0pt,partopsep=0pt, leftmargin=3.5mm]
                    \item Transparent
                    \item Good for non discrete values
                \end{itemize}
            &   \begin{itemize}[noitemsep,topsep=0pt,parsep=0pt,partopsep=0pt, leftmargin=3.5mm]
                    \item Inconsistent constraints
                    \item No results
                \end{itemize} \\ 
        \end{tabularx}
        \caption{A description of the advantages and disadvantages of common recommendation techniques \cite{richthammerSituationAwarenessRecommender2018, shokeenStudyFeaturesSocial2019,hahslerRecommenderlabFrameworkDeveloping2015, aminiDiscoveringImpactKnowledge2011, suSurveyCollaborativeFiltering2009}}
        \label{tab:Foundations:RecommenderComparison}
    \end{center}
\end{table}

\subsection{Hybrid Recommendation}
A hybrid recommender combines different recommendation approaches to use the strengths of each individual one and to reduce effects of weaknesses \cite{burkeHybridRecommenderSystems2002}.

\section{Group Recommender System}
\label{sec:Foundations:GroupRecommenderSystem}

A group recommender system is a recommender system aimed at making recommendations for a group instead of a single user. To make recommendations group members preferences have to be aggregated. This can be done by either aggregating single user recommendations or by merging preferences of each user into a group preference model. Based on the resulting preference model recommendation strategies as described in \autoref{sec:Foundations:RecommenderSystem} can be used to generate recommendations \cite{jamesonRecommendationGroups2007}.

The strategy of aggregating predictions can be further divided into two strategies. \citeauthor{felfernigAlgorithmsGroupRecommendation2018} \cite{felfernigAlgorithmsGroupRecommendation2018} describes merging recommendations and "ranking of candidate items". Merging recommendations can be used when multiple possible solutions should be presented. The recommender picks $n$ recommendation from each user's individual recommendations and merges them into a list. The second approach is that each user's individual recommender ranks all items. The group member specific rankings can are aggregated to get a group ranking of items. Instead of ranking it is also possible to simply predict a users rating for an item.

The aggregation of preferences uses a merging strategy to combine the individual preferences into group preferences. This allows a group to change its preferences during the course of the decision without changing individual preferences.

Both the approach of merging preferences and the approach of using individual users rankings require some kind of aggregation strategy. This section presents three strategies: multiplication, average and least misery. The multiplication strategy multiplies preferences of users and thereby combines them into a group preference. Similarly the average strategy takes the average of a rating and the least misery strategy takes the lowest rating among group members. To illustrate the example in \autoref{tab:Foundations:RecommenderSystem:MoviePreferences} is used. A group is formed out of Lucy, Eric and Diane. The resulting ratings for each strategy are shown in \autoref{tab:Foundations:RecommenderSystem:AggregationStrategy}.

\begin{table}
    \centering    
    \begin{tabular}{ l | c | c | c | c | c }
        & The Matrix & Titanic & Die Hard & Forest Gump & Wall-E \\ \hline
         multiplication  & 8 & - & 30 & 75 & - \\
         average  & $\frac{7}{3}$ & - & $\frac{10}{3}$ & $\frac{13}{3}$ & - \\
         least misery  & 1 & - & 2 & 3 & - \\
    \end{tabular}
    \caption{An example showing preference aggregation strategies for a group using the data from \autoref{tab:Foundations:RecommenderSystem:MoviePreferences}. Titanic and Wall-E were left out because not all group members have rated these movies.}
    
    \label{tab:Foundations:RecommenderSystem:AggregationStrategy}
\end{table}


\section{Base Recommender System}
\label{sec:Foundations:BaseSystem}

\citeauthor{raabKollaborativeProduktkonfigurationEchtzeit2019}'s \cite{raabKollaborativeProduktkonfigurationEchtzeit2019} extends CAS Merlin Configurator in his thesis to allow simultaneous configuration. The extended architecture is shown in \autoref{fig:DesignImplementation:CollaborativeConfiguratorMerlin}.
He only makes changes to M.Customer which is renamed to M.Collab-Customer and introduces a new component M.Collab.

\begin{figure}
    \centering
    \includegraphics[width=0.6\textwidth]{./figures/50_design_and_implementation/MerlinCollaborativeConfigurator.pdf}
    \caption{Architecture of Collaborative Configurator Merlin \cite[Fig. 4.3]{raabKollaborativeProduktkonfigurationEchtzeit2019}}
    \label{fig:DesignImplementation:CollaborativeConfiguratorMerlin}
\end{figure}

\begin{description}
    \item[M.Core] provides the base of the configurator. It checks the configuration against all rules in the database, provides possible alternatives if a change invalidates other parts of a configuration. The system relies on a CSP solver for valida tion and suggestion of alternatives.
    \item[M.Model] is the editor to create products and rules. These rules can then be uploaded to M.Core.
    \item[M.Collab] is a node.js server application that communicates with M.Core via REST-API and with M.Collab-Customer via WebSocket. It sits in between M.Collab-Customer and M.Core and handles all processing regarding collaborative configuration.
    \item[M.Collab-Customer] a modified version of M.Customer that does all communication via WebSocket and does communicate with M.Collab instead of M.Core. M.Customer is the customer facing component. It allows a customer to configure a product or solution.
\end{description}

\FloatBarrier


