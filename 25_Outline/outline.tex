\documentclass{article}
\usepackage{tabularx}
\usepackage{placeins}

\newcommand{\tabitem}[1][\textbullet]{~~\llap{#1}~~}

\begin{document}
\title{Collaborative Decision-Making in Group-Based Configuration Processes}
\author{Hannes F. Kuchelmeister}
\date{2020/01/23}

\maketitle

\section{Research Gap}

Collaborative Filtering for group recommender and configuration recommender but not 

\begin{center}
    \begin{tabular}{ p{5cm}|p{5cm} } 
        \hline
        Group Recommender & Configuration Recommender \\
        \hline
        Preferences & Current unfinished configuration \\ 
        Aggregation Strategy (aggregate preferences vs. aggregate recommendations &  \\
        Use aggregation strategy for recommendations or preferences to decide on which product will be choosen (e.g Song) & Compare current unfinished configuration with stored finished configuration. \\
        \hline
        \multicolumn{2}{c}{This thesis} \\
        \hline
        \multicolumn{2}{c}{aggregate preferences} \\
        \multicolumn{2}{c}{current unfinished configuration combined with preferences} \\
    \end{tabular}
\end{center}

\section{Problem}
A group of people with different personal preferences (knowledge and hierarchical power) want to buy products with high variability. Making decisions in the group comes with problems as communicating preferences is complicated.

Examples of that are:
\begin{itemize}
    \item A companies truck fleet
    \item A companies customer management software system
    \item A public project to get a new area at a zoo
    \item Managing a forest (while considering public interests and also wood production companies)
    \item An existing building and how it should be furnished
\end{itemize}. 

\section{Solution Objectives}

\begin{itemize}
    \item A system should give recommendation for the group using a utility function that takes into account preferences of group members, the current state (and potentially hierarchy and knowledge)  of group members.
    \item Recommendations should allow different strategies
\end{itemize}

\begin{itemize}
    \item Recommendations should always be valid options (i.e. configurations)
    \item Recommendations should allow different strategies
\end{itemize}

\FloatBarrier

\section{Use Cases}
\begin{table}
    \begin{center}
        \begin{tabularx}{\columnwidth}{l|X}
            \multicolumn{2}{c}{Main Use Case} \\
            \hline
            Preconditions   & \tabitem The configurator is opened with the same session on each of the group member's machines \\
                            & \tabitem The configuration is in an unfinished state (this state is a consensus state)\\
            \hline
            Postcondition   & \tabitem All users have given their preferences as ratings of attributes \\
                            & \tabitem The system gives a recommendation based on all preferences and the unfinished configuration state \\
            \hline
            Basic Flow      & \tabitem[1.] A user rates an attribute \\
                            & \tabitem[2.] The system generates a recommendation (based on preferences and configuration status) \\
                            & \tabitem[3.] If not all users have given their preferences go to step 1. \\
            \hline
        \end{tabularx}
        \caption{Simulation parameters}
        \label{table: simulation parameters}
    \end{center}
\end{table}

\FloatBarrier

\section{Technical Focus}

Given an unfinished configuration and preferences of all group members rate a finished configuration on how well "similar" it reflects the configuration + preferences.

$$utility : (configurationState,\ preferences,\ configurationToRate) \mapsto rating$$

Use this to choose the best finished configuration out of a list to recommend

\section{Possible extensions}

\begin{itemize}
    \item How to optimise such that no need to search through all stored finished configurations is necessary? (e.g. improve runtime from $\mathcal{O}(n)$ to $\mathcal{O}(log\ n)$)
    \item How to model hierarchy and knowledge about part of the products in preferences?
\end{itemize}
    
    
\section{Evaluation}

For one example e.g. forest example generate all possible valid configurations.

Generate groups with preferences (explicit preferences) and configuration state (which would be for example the currently existing forest).

\begin{itemize}
    \item Groups shall be generated with random preferences
    \item With grouped preferences -> people adhere more or less to one profile (Forest Owner, Athlete, Consumer, Environmentalist)
    \item Group of only one profile type -> rather homogenous group
\end{itemize}

With these preferences analyse how close the recommendations are to ideal recommendations depending on the number of stored finished configurations.

Is this approach practical -> how many finished configurations do you need to get good recommendations.


\end{document}