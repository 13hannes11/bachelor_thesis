\chapter{Concept}

\section{User Interaction with the System}
The system has one main way to be used as defined in \autoref{tab:MainUseCase}.

\begin{figure}
    \centering
    \includegraphics[width=1\textwidth]{./figures/bpmn_configuration_process_with_continious_recommendation.pdf}
    \caption{A bpmn diagram of the configuration process.}
    \label{fig:ConfigurationProcess}
\end{figure}

\begin{table}
    \begin{center}
        \begin{tabularx}{\columnwidth}{l|X}
            \multicolumn{2}{c}{Main System Usage} \\
            \hline
            Preconditions   & 
                \begin{itemize}
                    \item The configurator is opened with the same session on each of the group member's machines
                    \item The configuration is in an unfinished state (this state is a consensus state)
                \end{itemize} \\
            \hline
            Postcondition   & 
                \begin{itemize}
                    \item All users have entered their preferences for each attribute explicitly.
                    \item The system gives a recommendation based on all preferences and the unfinished configuration state
                \end{itemize} \\
            \hline
            Basic Flow      & 
                \begin{enumerate}
                    \item A user indicates a preference for an attribute
                    \item The system generates a recommendation (based on preferences and configuration status)
                    \item If not all users have given their preferences go to step 1.
                \end{enumerate} \\
            \hline
        \end{tabularx}
        \caption{A description of the main way users will interact with the system}
        \label{tab:MainUseCase}
    \end{center}
\end{table}

\FloatBarrier

\section{Solution Objectives}

Given an unfinished configuration and preferences of all group members rate a finished configuration on how well "similar" it reflects the configuration and preferences.

Use this to choose the best finished configuration out of a list to recommend.

\subsection{Generating a Recommendation}

Hereby the idea is there is a database of complete configurations (possibly historic from other groups or automatically generated or both).
Now the recommendation procedure looks as follows:

\begin{enumerate}
    \item Assign a score to each stored configuration according to $$score_{group}(\overline{configurationState},\ \overline{preferences}, \ configurationInStore)$$
    \item Chose the configuration with the highest score as recommendation.
\end{enumerate}



\section{Benefits}

The benefits of this approach are, that the need for a group to communicate is reduced. Each user gives their own preferences and the group will get a recommendation based on that. This allows to reduce problems with communication of preferences and eliminates misunderstandings.
    
