\chapter{Introduction}
\label{ch:Introduction}

\section{Research Gap}
\label{sec:Introduction:ResearchGap}

There does not exists much research on recommendation for group configuration, however it is comprised of two different areas of research, recommenders for groups and recommenders for configuration.
The existing literature on recommenders for groups is extensive with many different approaches and domains \cite{delicResearchMethodsGroup2016, chenInterfaceInteractionDesign2011, atasItemRecommendationUsing2017, jamesonRecommendationGroups2007, chenEmpatheticonsDesigningEmotion2014, liuCGSPAComprehensiveGroup2019}. \citeauthor{felfernigGroupRecommenderSystems2018} give an overview about basic approaches \cite{felfernigGroupRecommenderSystems2018}.
In the area of product configuration research about recommender systems is undertaken as well \cite{pereiraFeatureBasedPersonalizedRecommender2016, scholzConfigurationbasedRecommenderSystem2017, scholzEffectsDecisionSpace2017}.
Group configuration is a more specialized sub field of configuration therefore less research exists on that and because of that there has not been much research on recommenders that are for group recommendation in a configuration setting. 

%Commonly for content-based recommenders categories based on content are created and a separate user or group profile is generated based on the preferences of whole items. For configuration recommenders however this would create additional modelling or content grouping workload, therefore in this thesis it is proposed to use attributes of a configuration as distinguishing categories.

\section{Problem}
\label{sec:Introduction:Problem}

A group of people with different personal preferences wants to find a solution to a problem with high variability. Making decisions in the group comes with problems as a lack of communicating leads to worse decision outcomes \cite{atasItemRecommendationUsing2017}. Group dynamics and biases can lead to suboptimal decisions \cite{kerrBiasJudgmentComparing1996}. 

Examples of that are:
\begin{itemize}
    \item A companies truck fleet
    \item A companies customer management software system
    \item A public project to get a new area at a zoo
    \item Managing a forest (while considering public interests and also wood production companies)
    \item An existing building and how it should be furnished
\end{itemize}

\section{Solution Objectives}
\label{sec:Introduction:Solution Objectives}

\begin{itemize}
    \item A system should give recommendation for the group using a scoring function that takes into account preferences of group members and the current state of the situation.
    \item Recommendations should allow different scoring functions.
    \item Recommendations should always be valid options.
\end{itemize}