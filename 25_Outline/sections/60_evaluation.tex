\chapter{Evaluation}
\label{ch:Evaluation}


\section{Evaluation of Case Study}
\label{sec:Evaluation:CaseStudy}

For one example e.g. forest example generate all possible valid configurations.

Generate groups with preferences (explicit preferences) and configuration state (which would be for example the currently existing forest).

\subsection{Group Types During Evaluation}
\begin{itemize}
    \item Groups shall be generated with random preferences
    \item With grouped preferences: people adhere more or less to one profile (Forest Owner, Athlete, Consumer, Environmentalist)
    \item Group of only one profile type: rather homogenous group
\end{itemize}

\subsection{Questions to Answer During the Evaluation}

\begin{itemize}
    \item How close are recommendations of the recommender system to the ideal recommendation depending on the number of stored recommendations? The ideal configuration is the configuration that has the highest score with the given group scoring function $score_{group}$.
    \item Is this approach practical?
\end{itemize}