\chapter{Organisation}
\label{ch:Organisation}

This chapter provides an overview on time schedule, advisers and other organisational circumstances regarding this thesis.

\section{Advisers}
\label{sec:Organisation:Advisers}

\todo[inline]{Add section about advisors and organisational structure of CAS}

\section{Artefacts}
\label{sec:Organisation:Artefacts}

This thesis consist of multiple artefacts. Artefacts do not have to be separate documents but can be included in other documents. Those are:
\begin{itemize}
    \item Proposal (this document)
    \item Proposal presentation
    \item Bachelor's thesis
    \item Bachelor's thesis presentation
    \item Development Related Artefacts
    \begin{itemize}
        \item Design documentation
        \item Source code documentation
        \item Source code
        \item One or multiple sample configurations
        \item A working Prototype
    \end{itemize}
    \item Evaluation data
\end{itemize}

\section{Tools}
\label{sec:Organisation:Tools}

\todo[inline]{Add Tools}

\section{Process}
\label{sec:Organisation:Process}

A slightly reduced version of Scrum is used during the creation of this thesis. Scrum is the de-facto standard for agile software development \cite{glogerScrumPradigmenwechselIm2010}. Daily meetings with status updates and suggestions with advisory and colleagues at CAS are held every day. The Sprint length is one week but during later development this potentially will be lengthened to two weeks. Sprint reviews, retrospective and sprint planning are not explicitly done. Hereby the process is not used for coordination of a team that works on a single software project but to improve knowledge exchange and communication.

\section{Schedule}
\label{sec:Organisation:Schedule}

The planned schedule can be seen in \ref{fig:Organisation:Schedule:Gant}. Documentation and of the thesis is planned to be written in parallel to  \emph{Problem Identification}, \emph{Design and Implementation} and \emph{Demonstration and Evaluation} phases. After all these phases the thesis documentation will be finalized and corrected. 

\begin{figure}
    \begin{center}
        \begin{ganttchart}[y unit title=0.7cm,
            y unit chart=0.8cm, 
            title height=1,
            bar height=0.6,
            group right shift=0,
            group top shift=.6,
            group height=.3
            ]{1}{17}
            \gantttitle{Week}{17}\ganttnewline
            \gantttitlelist{1,...,17}{1}\ganttnewline
            \ganttgroup{Thesis}{1}{17}\ganttnewline
    
            \ganttgroup{Problem Identification}{1}{2}\ganttnewline
            \ganttbar{Literature Research}{1}{1}\ganttnewline
            \ganttbar{Defintion of Objectives}{1}{2}\ganttnewline
            
            
            \ganttgroup{Design and Implementation}{3}{10}\ganttnewline
            \ganttbar{Design}{3}{4}\ganttnewline
            \ganttbar{Implementation}{4}{9}\ganttnewline
            \ganttmilestone{Code Review}{8}\ganttnewline
            \ganttmilestone{Fully Working Prototype}{9}\ganttnewline
            \ganttbar{Buffer}{10}{10}\ganttnewline

            \ganttgroup{Demonstration and Evaluation}{11}{12}\ganttnewline
            \ganttbar{Demonstration}{11}{11}\ganttnewline
            \ganttbar{Evaluation}{12}{12}\ganttnewline

            \ganttgroup{Documentation}{1}{16}\ganttnewline
            \ganttbar{Write Documentation}{1}{12}\ganttnewline
            \ganttmilestone{First Draft}{12}\ganttnewline
            \ganttbar{Finalize Thesis}{13}{14}\ganttnewline
            \ganttmilestone{Final Draft}{14}\ganttnewline
            \ganttbar{Proof Reading and Corrections}{15}{16}\ganttnewline
            \ganttmilestone{Finished Thesis}{16}\ganttnewline
            \ganttbar{Buffer}{17}{17}\ganttnewline

            \ganttvrule{}{2}
            \ganttvrule{}{10}
            \ganttvrule{}{12}
            \ganttvrule{}{16}
        \end{ganttchart}
    \end{center}
    \caption{Gantt chart representing the schedule of this thesis.
    }
    \label{fig:Organisation:Schedule:Gant}
\end{figure}

\section{Risks Assessment}
\label{sec:Organisation:RiskAssessment}

\begin{enumerate}[label=Risk \arabic*:, align=left, leftmargin=*]
    \item Underestimation of effort for \emph{Design and Implementation} phase
        \begin{description}
            \item[Explanation:] Tasks might take more effort then expected and therefore cannot be completed in the planned time. This is especially problematic for the \emph{Design and Implementation} phase as the \emph{Demonstration and Evaluation} phase requires  a working prototype.
            \item[Mitigation:] There are planned explicit buffers. One for the \emph{Design and Implementation} phase and one general buffer. This amounts to two weeks of buffer. Additionally it is possible to shorten \emph{Finalize Thesis} and \emph{Proof Reading and Corrections} phases as the time planned for these phases is overestimated. Overall in a worst-case scenario it is possible to extend \emph{Design and Implementation} by up to four weeks.
        \end{description}
    
    \item Delay due to unforeseen circumstances outside of the authors control
        \begin{description}
            \item[Explanation:] Outside influence could delay the thesis. An example of that is sickness.
            \item[Mitigation:] According to "Prüfungsordnung" §14.6 it is possible to extend the period of the thesis of up to a month \cite[~ p. 724]{StudienUndPrufungsordnung2015}.
        \end{description}

    \item Bugs and architectural problems in prototype
        \begin{description}
            \item[Explanation:] Bugs in the prototype that this thesis will extend could delay implementation of features. Also architectural problems could make it harder then expected to extend the prototype.
            \item[Mitigation:] To reduce work in such a case less features could be implemented or some features could just use a mock implementation.
        \end{description}
\end{enumerate}