\chapter{Foundations}
\label{ch:Foundations}

\section{Group Decision-Making}
\label{sec:Foundation:GroupDecisionMaking}

\section{Decision Tasks}
\label{sec:Foundation:DecisionTask}

\section{Recommender System}
\label{sec:Foundation:RecommenderSystem}

\section{Single User Decision Scenario}
\label{sec:Foundation:SingleUserDecisionScenario}

\section{Group Decision Scenario}
\label{sec:Foundation:GroupDecisionScenario}

\section{Product Configuration}
\label{sec:Foundation:ProductConfiguration}

Product configuration is a process consisting of a series of decision tasks whereby a product is constructed of components which interact with each other. During a configuration process no new components are created their interplay and specification is defined beforehand \cite[~ p. 42-43]{sabinProductConfigurationFrameworksa1998}.

Formally a configuration problem can be specified as a \emph{constraint satisfaction problem (CSP)} \cite{tsangFoundationsConstraintSatisfaction1993} as 
\[
    CSP(V,D,C)
\]
where \( V = \{v_1,\dots, v_n\} \) is a set of variables, \( D = dom : V \mapsto X \) is a relation of variables and their corresponding domain definitions \( X \), and \( C = C_{PREF} \cup C_{KB} \) is a set of constraints with customer preferences \( C_{PREF} \) and configuration knowledge base \( C_{KB} \) \cite{felfernigOpenConfiguration2014, felferningGroupBasedConfiguration2016}.

\section{Collaborative Product Configuration}
\label{sec:Foundation:CollaborativeProductConfiguration}

