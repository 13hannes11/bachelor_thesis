\chapter{Foundations}
\label{ch:Foundations}

\section{Recommender System}
\label{sec:Foundations:RecommenderSystem}

A recommender system is a system that gives individualized recommendations to users to guide them through a large space of objects \cite[~ p. 331]{burkeHybridRecommenderSystems2002}.

There are several approaches to recommender systems presented in \cite{felfernigGroupRecommenderSystems2018}, these are: collaborative filtering, content-based filtering, critiquing-based filtering, constraint-based, hybrid recommendation.

\subsection{Collaborative Filtering}
In collaborative filtering a users rating for unknown items is predicted by finding similar users who have rated it. Their rating is used as prediction
\cite[~ pp. 7, 8]{felfernigDecisionTasksBasic2018}.

\subsection{Content-Based Filtering}
Items and users are assigned to categories. Based on consumption and rating of items a user will have implicit ratings for categories. Predictions are now made based on a categories of the new item \cite[~ pp. 10, 11]{felfernigDecisionTasksBasic2018}.

\subsection{Critiquing-Based Recommendation}
Items are recommended and a user can then critique on attributes of the recommendation. Based on that a similar item which does not have those critiques can be recommended. User preferences are implicitly collected this way \cite{knijnenburgEachHisOwn2011}.

\subsection{Constraint-Based Recommendation}
Hereby filter rules are defined which filter out items that don't fulfil specified rules. A user models their requirements with these rules and thereby gets a list of recommended items \cite[~ p. 12]{felfernigDecisionTasksBasic2018}

\subsection{Hybrid Recommendation}
A hybrid recommender combines different recommendation approaches to use the strengths of each individual one and to reduce effects of weaknesses \cite{burkeHybridRecommenderSystems2002}.

\todo[inline]{add more detailed description of recommendation systems - do they need historicle data  which conditions for usage are needed, strengths and weaknesses}

\section{Group Recommender System}

A group recommender system is a recommender system aimed at making recommendations for a group instead of a single user. To make recommendations group members preferences have to be aggregated. This can be done by either aggregating single user recommendations or by merging preferences of each user into a group preference model. Based on this model recommendation strategies as described in \ref{sec:Foundations:RecommenderSystem} can be used to generate recommendations \cite{jamesonRecommendationGroups2007}.

\section{Product Configuration}
\label{sec:Foundations:ProductConfiguration}

Product configuration is a process consisting of a series of decision tasks whereby a product is constructed of components which interact with each other. During a configuration process no new components are created. Their interplay and specification is defined beforehand \cite[~ pp. 42, 43]{sabinProductConfigurationFrameworksa1998}.

Formally a configuration problem can be specified as a \emph{constraint satisfaction problem (CSP)} \cite{tsangFoundationsConstraintSatisfaction1993} as 
\[
    CSP(V,D,C)
\]
where \( V = \{v_1,\dots, v_n\} \) is a set of variables, \( D = dom : V \mapsto X \) is a relation of variables and their corresponding domain definitions \( X \), and \( C = C_{PREF} \cup C_{KB} \) is a set of constraints with customer preferences \( C_{PREF} \) and configuration knowledge base \( C_{KB} \) \cite{felferningGroupBasedConfiguration2016, felfernigOpenConfiguration2014}.


\section{Group-Based Product Configuration}
\label{sec:Foundations:GroupBasedProductConfiguration}

To define group-based product configuration we extend the definition (\ref{sec:Foundations:ProductConfiguration}) to 
\[ 
    C_{PREF} = \bigcup 
PREF_i \]
with preferences of user \( i \) as \( PREF_i \) \cite{ felferningGroupBasedConfiguration2016}.

\section{Group-Based Configuration-Solution}
\label{sec:Foundations:GroupBasedConfigurationSolution}

\ref{sec:Foundations:ProductConfiguration} and \ref{sec:Foundations:GroupBasedProductConfiguration} expand to a solution of a group-based configuration with the addition of variable assignments
\[
    C_{CONF} = \bigcup_{v_i \in V} \{ v_i = x_i \}, \ x_i \in dom(v_i)
\]
and where \( C_{CONF} \cup C_{PREF} \cup C_{KB} \) is consistent \cite{ felferningGroupBasedConfiguration2016}.

\todo[inline]{add group descision part, conflict resolution, types of conflict, differences in knowledge/decision power;

group dynamics (e.g. biases in decision making, anchoring, etc.)

Why are group decision interesting?}



