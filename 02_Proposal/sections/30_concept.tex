\chapter{Concept}
\label{ch:Concept}

\section{CAS Configurator Merlin}
\label{sec:Concept:ConfiguratorMerlin}

\ref{fig:Concept:ConfiguratorMerlin} shows the architecture of CAS Configurator Merlin.
\begin{description}
    \item[M.Core] provides the base of the configurator it checks configuration against all rules in the database, provides possible alternatives on a change that invalidates other parts of a configuration.
    \item[M.Model] is the editor to create products and rules. These rules can then be uploaded to M.Core.
    \item[M.Customer] is the customer facing component. It allows a customer to configure a product.
\end{description}

\begin{figure}
    \centering
    \includegraphics{./figures/MerlinConfigurator.pdf}
    \caption{Architecture of Configurator Merlin \cite[Fig. 4.1]{raabKollaborativeProduktkonfigurationEchtzeit2019}}
    \label{fig:Concept:ConfiguratorMerlin}
\end{figure}

\section{CAS Group-Configurator}
\label{sec:Concept:GroupConfigurator}


\citeauthor{raabKollaborativeProduktkonfigurationEchtzeit2019} extends CAS Merlin Configurator in his thesis to allow simultaneous configuration. The extended architecture is shown in Figure \ref{fig:Concept:CollaborativeConfiguratorMerlin}.
He only makes changes to M.Customer which is renamed to M.Collab-Customer and introduces a new component M.Collab.

\begin{description}
    \item[M.Collab] is a node.js server application that communicates with M.Core via REST-API and with M.Collab-Customer via WebSocket. It sits in between M.Collab-Customer and M.Core and handles all processing regarding collaborative configuration.
    \item[M.Collab-Customer] a modified version of M.Customer that does all communication via WebSocket and does communicate with M.Collab instead of M.Core.
\end{description}

\begin{figure}
    \centering
    \includegraphics{./figures/MerlinCollaborativeConfigurator.pdf}
    \caption{Architecture of Collaborative Configurator Merlin \cite[Fig. 4.3]{raabKollaborativeProduktkonfigurationEchtzeit2019}}
    \label{fig:Concept:CollaborativeConfiguratorMerlin}
\end{figure}

\section{Extended Configurator}
\label{sec:Concept:ExtendedConfigurator}
\missingfigure{Extenden Architecture}